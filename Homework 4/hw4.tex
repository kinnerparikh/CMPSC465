\documentclass{article} % This command is used to set the type of document you are working on such as an article, book, or presenation

\usepackage{geometry} % This package allows the editing of the page layout
\usepackage{amsmath}  % This package allows the use of a large range of mathematical formula, commands, and symbols
\usepackage{graphicx}  % This package allows the importing of images
\usepackage{soul}
\usepackage{amsfonts}
\usepackage{dirtytalk}
\usepackage{tabto}
\usepackage{xcolor,colortbl, amssymb}
\usepackage{forest}
\usepackage[ruled, lined, linesnumbered, commentsnumbered, longend]{algorithm2e}

% https://www.messletters.com/en/big-text/

\newcommand{\question}[2][]{\begin{flushleft}
        \textbf{Problem #1}: \textit{#2}

\end{flushleft}}

\definecolor{Green}{rgb}{0, 1, 0}
\definecolor{Pink}{rgb}{1, .753, .796}

\newcommand{\sol}{\textbf{Solution}:} %Use if you want a boldface solution line
%\newcommand\tab[1][0.4cm]{\hspace*{#1}}
\newcommand{\maketitletwo}[2][]{\begin{center}
        \Large{\textbf{Homework #1}
            
            CMPSC 465} % Name of course here
        \vspace{5pt}
        
        \normalsize{Kinner Parikh  % Your name here
        
        \today}        % Change to due date if preferred
        \vspace{40pt}


        \newpage
        
\end{center}}
\begin{document}
    \maketitletwo[4]  % Optional argument is assignment number
    %Keep a blank space between maketitletwo and \question[1]

    \question[1]{}
    \begin{center}
        
        I did not work in a group
    
        I did not consult without anyone my group member
    
        I did not consult any non-class materials
    \end{center}
    
    \newpage

    \question[2]{Divide and Conquer}

    a) Pseudo-code

    \begin{algorithm}
        \SetKwFunction{findMajorityElement}{findMajorityElement}
        \SetKwInOut{KwIn}{Input}
        \SetKwInOut{KwOut}{Output}

        \KwIn{An array $A[1 \ldots n]$ that may or may not contain a majority element (an element for which appears more than $n/2$ times in the array) }
        \KwOut{The majority element}
        $majorityElement= A[1],\ count = 1$

        \For{$i \leftarrow 1$ \KwTo $n$}{
            \uIf{count = 0}{
                $majorityElement :e= A[i]$

                $count \leftarrow 1$
            }
            \uElseIf{$majorityElement = A[i]$}{
                $count := count + 1$
            }
            \Else(){
                $count := count - 1$
            }
        }

        \Return{$majorityElement$}

        \caption{Finding majority element in an array}
    \end{algorithm}



    \newpage

    \question[3]{Reverse Graph}

    \newpage

    \question[4]{Graph Basics}



\end{document}
