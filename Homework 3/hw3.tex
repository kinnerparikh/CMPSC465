\documentclass{article} % This command is used to set the type of document you are working on such as an article, book, or presenation

\usepackage{geometry} % This package allows the editing of the page layout
\usepackage{amsmath}  % This package allows the use of a large range of mathematical formula, commands, and symbols
\usepackage{graphicx}  % This package allows the importing of images
\usepackage{soul}
\usepackage{amsfonts}
\usepackage{dirtytalk}
\usepackage{tabto}
\usepackage{xcolor,colortbl, amssymb}
\usepackage{forest}
\usepackage{comment}

% https://www.messletters.com/en/big-text/

\newcommand{\question}[2][]{\begin{flushleft}
        \textbf{Problem #1}: \textit{#2}

\end{flushleft}}

\definecolor{Green}{rgb}{0, 1, 0}
\definecolor{Pink}{rgb}{1, .753, .796}

\newcommand{\sol}{\textbf{Solution}:} %Use if you want a boldface solution line
%\newcommand\tab[1][0.4cm]{\hspace*{#1}}
\newcommand{\maketitletwo}[2][]{\begin{center}
        \Large{\textbf{Homework #1}
            
            CMPSC 465} % Name of course here
        \vspace{5pt}
        
        \normalsize{Kinner Parikh  % Your name here
        
        \today}        % Change to due date if preferred
        \vspace{40pt}


        \newpage
        
\end{center}}
\begin{document}
    \maketitletwo[3]  % Optional argument is assignment number
    %Keep a blank space between maketitletwo and \question[1]

    \question[1]{}
    \begin{center}
        
        I did not work in a group
    
        I did not consult without anyone my group member
    
        I did not consult any non-class materials
    \end{center}
    
    \newpage

    \question[2]{Solving recurrences}

    a) $T(n) = 11T(n / 5) + 13n^{1.3}$

    % $a = 11, b = 5, d = 1.3$

    % $\Theta(13n^{1.3})$

    $W_k = 11^k \times 13(n/5^k)^{1.3}$

    $\sum_{k = 0}^{\log_5n} W_k = 13n^{1.3} \sum_{k = 0}^{\log_5n} \left(\frac{11}{5^{1.3}}\right)^k = \Theta(13n^{1.3} \cdot \left(\frac{11}{5^{1.3}}\right)^{\log_5n}) = \boxed{\Theta(13n^{\log_511})}$

    \vspace{5pt}

    b) $T(n) = 6T(n / 2) + n^{2.8}$

    $ a = 6, b = 2, d = 2.8$

    $\log_26 < 2.8$, so by Master's theorem, $\boxed{\Theta(n^{2.8})}$
    % $W_k = 6^k \times (n/2^k)2.8$

    \vspace{5pt}

    c) $T(n) = 5T(n / 3) + \log^2n$

    %$W_k = 5^k \times \log^2(n/3^k)$

    %$\sum_{k = 0}^{\log_3n} W_k = \sum_{k = 0}^{\log_3n} 5^k \times (\log^2n - k\log^2(3)) = \Theta(5^{\log_3n} \cdot (\log^2n - \log_3n \cdot \log^23))$

    Lower Bound Case: $1 < \log^2n$ and $\log_35 > 1$

    Upper Bound Case: $\log^2n < n$ and $\log_35 > n$

    Thus, $\boxed{\Theta(n^{\log_35})}$.

    \vspace{5pt}

    d) $T(n) = T(n - 2) + \log n$

    This will recurse $n/2$ times so total work will be $\sum_{k = 1}^{n/2}\log(2k)$.

    This will be $\log(2) + \log(4) + ... + \log(\frac{n}{2} - 1) + \log(\frac{n}{2})$, this simplifies to $\log(n^n)$, thus $\boxed{\Theta(n\log n)}$.

    \newpage

    \question[3]{Sorted Array}

    The basic structure for this algorithm is that you take an upper($u$) and lower($l$) bound of indeces 
    and find the average($\texttt{avg} = \frac{u + l}{2})$ to check if the value at that index is equal to, greater than, or less 
    than the index. If it is equal to the index, then we have found our $A[i] = i$. If $i < A[i]$, then we make a 
    recursive call with $l$ staying as is, but $u = \texttt{avg}$. If $i > A[i]$, then we make a recursive call 
    with $l = \texttt{avg}$ and $u$ staying as is. This will eventually converge to find the $A[i] = i$ in $O(\log n)$ 
    time. This cuts the problem in half every single time because we are changing the bounds for which we 
    search. For example, suppose we are given array: \texttt{[-2, -1, 1, 3, 6, 7, 10, 11]}. This will 
    be broken down into this tree:

    \begin{center}
        \begin{forest}
            [\texttt{[-2, -1, 1, 3, \hl{6}, 7, 10, 11]}
                [\texttt{[-2, -1, \hl{1}, 3]}
                    [\texttt{[-2, \hl{-1}]}[\texttt{[-2]}][\texttt{[-1]}]]
                    [\texttt{[\hl{1}, 3]}[\texttt{[1]}][\texttt{[3]}]]
                ] 
                [\texttt{[6, \hl{7}, 10, 11]}
                    [\texttt{[6, \hl{7}]} [\texttt{[6]}][\texttt{[7]}]]
                    [\texttt{[\hl{10}, 11]} [\texttt{[10]}][\texttt{[11]}]]
                ]
            ]
        \end{forest}
    \end{center}

    The highlighted numbers are those that will be compared to their index in the array. If the 
    index is less than the value, it continues down the left side of the tree, and the opposite for if the 
    index is greater than the value. Ultimately, this leads us to see that $T(n) = 2T(n/2) + 1$. Using 
    Master's Theorem, we can see that $ a = 2, b = 2, d = 0$, which shows us that this will run in $O(\log n)$ time. 
    
    % [-2, -1, 2, 3, 6, 7, 10]
    %   0   1  2  3  4  5  6

    \newpage
    
    \question[4]{Linear Time Sorting}

    % Let us assume that the first of $x$ is 1. The algorithm is extremely straightforward, if $x[i] = i$, 
    % then move on, but if it isn't, swap $x[i]$ and $x[x[i]]$. For example, if $i = 3$, then $x[i] = 6$, then whatever value is 
    % at $x[6] (x[x[i]])$ will be swapped into $x[i]$ and the original $x[i]$ will be placed in 
    % $x[6]$. We can iterate through the array once and perform these operations, and we will get a sorted
    % array of integers which is, in essence, $O(n)$. The reason for the $M$ is that there could be a value which 

    The algorithm for this is fairly straightforward. We iterate through the array once to find 
    the largest and smallest value in the array, let us call these $x_{max}$ and $x_{min}$. We can then create $m$ 
    number of buckets, one for each integer between $x_{max}$ and $x_{min}$, where $m = x_{max} - x_{min}$. Then 
    passing through the array one more time, we can place the values into their respective buckets. 
    Finally, from these buckets, we can place the values into $x$ in order, and we can say that $x$ is sorted.
    As an example, take the array:
    \begin{center}
        $x$ = \texttt{[4, 9, 1, 4, 2, 0, 10, 2, 5, 2, 1, 8, 2, 6, 2, 5]}
    \end{center}
    We can do one pass through the array and see that $x_{min} = 0$ and $x_{max} = 10$. Then we can make 
    buckets for all the values between 0 and 10. Running through the array one more time, we can 
    place the number into its bucket. This would look like:
    \newline
        \texttt{
            0 $\rightarrow$ [0] \newline
            1 $\rightarrow$ [1, 1] \newline
            2 $\rightarrow$ [2, 2, 2, 2, 2] \newline
            3 $\rightarrow$ [] \newline
            4 $\rightarrow$ [4, 4] \newline
            5 $\rightarrow$ [5, 5] \newline
            6 $\rightarrow$ [6] \newline
            7 $\rightarrow$ [] \newline
            8 $\rightarrow$ [8] \newline
            9 $\rightarrow$ [9] \newline
            10 $\rightarrow$ [10] \newline}
    Then we can combine these in order to get:
    \begin{center}
        $x$ = \texttt{[0, 1, 1, 2, 2, 2, 2, 2, 4, 4, 5, 5, 6, 8, 9, 10]}
    \end{center}
    This will run in $O(n + m)$ because we pass through the array twice, which is $O(n)$, and then adding 
    it from the buckets will take $O(m)$ time, so $O(n) + O(m) = O(n + m)$.

\end{document}
