\documentclass{article} % This command is used to set the type of document you are working on such as an article, book, or presenation

\usepackage{geometry} % This package allows the editing of the page layout
\usepackage{amsmath}  % This package allows the use of a large range of mathematical formula, commands, and symbols
\usepackage{graphicx}  % This package allows the importing of images
\usepackage{soul}
\usepackage{amsfonts}
\usepackage{dirtytalk}
\usepackage{tabto}
\usepackage{xcolor,colortbl, amssymb}
\usepackage{forest}
\usepackage{comment}

% https://www.messletters.com/en/big-text/

\newcommand{\question}[2][]{\begin{flushleft}
        \textbf{Problem #1}: \textit{#2}

\end{flushleft}}

\definecolor{Green}{rgb}{0, 1, 0}
\definecolor{Pink}{rgb}{1, .753, .796}

\newcommand{\sol}{\textbf{Solution}:} %Use if you want a boldface solution line
%\newcommand\tab[1][0.4cm]{\hspace*{#1}}
\newcommand{\maketitletwo}[2][]{\begin{center}
        \Large{\textbf{Homework #1}
            
            CMPSC 465} % Name of course here
        \vspace{5pt}
        
        \normalsize{Kinner Parikh  % Your name here
        
        \today}        % Change to due date if preferred
        \vspace{40pt}


        \newpage
        
\end{center}}
\begin{document}
    \maketitletwo[1]  % Optional argument is assignment number
    %Keep a blank space between maketitletwo and \question[1]

    \question[1]{}
    \begin{center}
        
        I did not work in a group
    
        I did not consult without anyone my group member
    
        I did not consult any non-class materials
    \end{center}
    
    \newpage

    \question[2]{Analyzing run time.}

    1. $\sum^{n}_{i = 1} \frac{n-i}{5} = \frac{1}{5}\sum^{n}_{i = 1} n - i = \frac{1}{5} \cdot (\sum^{n}_{i = 1} n  - \sum^{n}_{i = 1} i) = \frac{1}{5} \cdot (n^2 - \frac{n^2 + n}{2}) = \frac{n^2 - n}{10} = \underline{\Theta(n^2)}$

    \vspace*{10 pt}

    2. $\sum^{n/4}_{i = 1} n - 4i = \sum^{n/4}_{i = 1} n - \sum^{n/4}_{i = 1} 4i = \Theta(n^2) + \Theta(4 \cdot \frac{n/4 (n/4 + 1)}{2}) = \Theta(\frac{33n^2 + 4n}{32}) = \underline{\Theta(n^2)}$

    \vspace*{10pt}

    3. $\lfloor \log n \rfloor + \sum^{\lfloor \log n \rfloor}_{i = 1} \frac{n}{2^i} = \lfloor \log n \rfloor + n \cdot \sum^{\lfloor \log n \rfloor}_{i = 1} 2^{-i} = \underline{\Theta(n)}$

    \newpage

    \question[3]{Polynomials and Horner's Rule}

    a) $\sum^{n}_{i = 0} a_i \cdot x_0^i$
    
    \hspace{12pt}Number of Multiplications = $\frac{n(n + 1)}{2}$ because each power $x^n_0$ is $n - 1$ multiplications, and you 
    
    \hspace{12pt}add one for multiplying $a_i$.
    
    \hspace{12pt}Number of Sums = $n$

    \vspace{5pt}
    

    b) LI = $\sum^{n - i}_{j = 0} a_{n - j}x^{n - i - j}$

    \vspace{5pt}

    \hspace{10pt} \textbf{Initialization:} At the start of the 1st iteration where $i = n$, the loop invariant states that 
    
    \hspace{10pt} $z = a_n$. Since this represents the coefficient for a polynomial with a maximum coefficient of 
    
    \hspace{10pt} 0, this is true.

    \vspace{5pt}
 
    \hspace{12pt}\textbf{Maintenance:} Assume that LI holds at the start of iteration $k_0$. This means that the 
    
    \hspace{12pt}polynomial is $a_{k_0} + (a_{k_0 + 1} x)  + ... + (a_{n - 1}x^{n - k_0 - 1}) + (a_nx^{n - k_0})$. We need to show that at 
    
    \hspace{12pt}iteration $k_0 + 1$, the LI consists of the $k$ largest polynomial powers of $x$ with their corresponsing 
    
    \hspace{12pt}coefficients. In the next loop, we get $k = k_0 - 1$ where 
    
    \hspace{2.5cm}$z = a_{k_0 - 1} + (a_{k_0} x)  + ... + (a_{n - 1}x^{n - k_0}) + (a_nx^{n - k_0 + 1})$ 
    
    \hspace{12pt}simplifies to: $z = a_{k} + (a_{k + 1} x)  + ... + (a_{n - 1}x^{n - k - 1}) + (a_nx^{n - k})$ % simply just add $a_{n - j - 1} x^{n-j}$ to the function stated above, which, 
    
    \hspace{12pt}Therefore we can say that the LI holds for every iteration of the algorithm.


    \vspace{5pt}

    \hspace{12pt}\textbf{Termination:} We must argue that the fact that the LI holds at the start of iteration $n$ 
    
    \hspace{10pt} implies the algorithm correct. When the algorithm stops, $i = 0$ so 
    
    \hspace{10pt} $z = a_0 + a_1x + ... + a_{n - 1}x^{n - 1} + a_nx^n$. 

    \vspace{7pt}

    


    c) Number of Sums = $n$

    \hspace{12pt}Number of Multiplications = $2n - 1$

    \newpage

    \question[4]{Solving recurrences}

    (a) $T(n) = 2T(n / 2) + \sqrt{n}$
    
    \hspace{13pt} Branching Factor = 2
    
    \hspace{13pt} Height = $\log_2n $

    \hspace{13pt} Size of subproblems at depth k = $n/2^k$

    \hspace{13pt} Number of subproblems at depth k = $2^k$

    \hspace{13pt} $W_k = 2^k \sqrt{n/2^k}$

    \hspace{13pt} $\sum^{\log_2 n}_{k = 0} W_k = \sum^{\log_2 n}_{k = 0} 2^k \sqrt{n/2^k} = \Theta(2^{\log_2n} \cdot \sqrt{n/(2^{\log_2n})}) = \underline{\Theta(n)}$

    \vspace{5pt}

    (b) $T(n) = 2T(n / 3) + 1$

    \hspace{13pt} Branching Factor = 2
    
    \hspace{13pt} Height = $\log_3n $

    \hspace{13pt} Size of subproblems at depth k = $n / 3^k$
    
    \hspace{13pt} Number of subproblems at depth k = $2^k$

    \hspace{13pt} $W_k = 2^k$

    \hspace{13pt} $\sum^{\log_3 n}_{k = 0} W_k = \sum^{\log_3 n}_{k = 0} 2^k = \underline{\Theta(n^{\log_3 2})}$
    
    \vspace{5pt}

    (c) $T(n) = 5T(n / 4) + n$

    \hspace{13pt} Branching Factor = 5
    
    \hspace{13pt} Height = $\log_4n $

    \hspace{13pt} Size of subproblems at depth k = $n / 4^k$
    
    \hspace{13pt} Number of subproblems at depth k = $5^k$

    \hspace{13pt} $W_k = 5^k \cdot n/4^k) = (5/4)^k \cdot n$

    \hspace{13pt} $\sum^{\log_4 n}_{k = 0} W_k = n \cdot \sum^{\log_4 n}_{k = 0} \frac{5}{4}^k = \underline{\Theta(n^{\log_4 5})}$

    \vspace{5pt} 

    (d) $T(n) = 7T(n / 7) + n$

    \hspace{13pt} Branching Factor = 7
    
    \hspace{13pt} Height = $\log_7n $

    \hspace{13pt} Size of subproblems at depth k = $n / 7^k$
    
    \hspace{13pt} Number of subproblems at depth k = $7^k$

    \hspace{13pt} $W_k = 7^k \Theta(n/7^k) = \Theta(n)$

    \hspace{13pt} $\sum^{\log_7 n}_{k = 0} W_k = \sum^{\log_7 n}_{k = 0} \Theta(n) = \underline{\Theta(n \cdot \log_7n)}$

    \vspace{5pt}

    (e) $T(n) = 9T(n / 3) + n^2$

    \hspace{13pt} Branching Factor = 9
    
    \hspace{13pt} Height = $\log_3n $

    \hspace{13pt} Size of subproblems at depth k = $n / 3^k$
    
    \hspace{13pt} Number of subproblems at depth k = $9^k$

    \hspace{13pt} $W_k = 9^k \cdot (n/3^k)^2 = n^2$

    \hspace{13pt} $\sum^{\log_3 n}_{k = 0} W_k = \sum^{\log_3 n}_{k = 0} n^2 = \underline{\Theta(n^2 \cdot \log_3n)}$

\end{document}
