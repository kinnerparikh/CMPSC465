\documentclass{article} % This command is used to set the type of document you are working on such as an article, book, or presenation

\usepackage{geometry} % This package allows the editing of the page layout
\usepackage{amsmath}  % This package allows the use of a large range of mathematical formula, commands, and symbols
\usepackage{graphicx}  % This package allows the importing of images
\usepackage{soul}
\usepackage{amsfonts}
\usepackage{dirtytalk}
\usepackage{tabto}
\usepackage{xcolor,colortbl, amssymb}
\usepackage{forest}
\usepackage[ruled, lined, linesnumbered, commentsnumbered, longend]{algorithm2e}
\usepackage{tikz}

% https://www.messletters.com/en/big-text/

\newcommand{\question}[2][]{\begin{flushleft}
        \textbf{Problem #1}: \textit{#2}

\end{flushleft}}

\definecolor{Green}{rgb}{0, 1, 0}
\definecolor{Pink}{rgb}{1, .753, .796}

\newcommand{\sol}{\textbf{Solution}:} %Use if you want a boldface solution line
%\newcommand\tab[1][0.4cm]{\hspace*{#1}}
\newcommand{\maketitletwo}[2][]{\begin{center}
        \Large{\textbf{Homework #1}
            
            CMPSC 465} % Name of course here
        \vspace{5pt}
        
        \normalsize{Kinner Parikh  % Your name here
        
        \today}        % Change to due date if preferred
        \vspace{40pt}


        \newpage
        
\end{center}}
\begin{document}
    \maketitletwo[9]  % Optional argument is assignment number
    %Keep a blank space between maketitletwo and \question[1]

    \question[1]{}
    \begin{center}
        
        I worked with Sahil Kuwadia and Ethan Yeung
    
        I did not consult without anyone my group member
    
        I did not consult any non-class materials
    \end{center}
    
    \newpage

    \question[2]{Suppose we have an optimal prefix code on a set $C = \{0,1,...,n-1\}$ of characters and we wish to transmit this code using as few bits as possible. Show how to represent any optimal prefix code on C using only $2n-1 + n\lceil \log n \rceil$ bits.}

    Since there are $n$ characters, there are $n$ leaves in the tree. Thus, there will be $n-1$ vertices within the graph, so the entire graph will contain $2n-1$ total vertices, thus $2n-1$ bits. The height of a full binary tree for $n$ characters is $\lceil \log n \rceil$. 

    We can say that to associate the members of C with the leaves of the tree, $\lceil \log n \rceil$ bits will be enough to represent all members. We know that no padding bits are required because each character is represented by a unique prefix. So, with $n$ leaves, it requires $n \lceil \log n \rceil$ bits to represent all characters.
    
    Thus, we can say that the total number of bits required to represent the optimal prefix code is $2n-1 + n\lceil \log n \rceil$.

    \newpage

    \question[3]{Generalize Huffman's algorithm to ternary codewords (i.e., codewords using the symbols 0, 1, and 2), and prove that it yields optimal ternary codes.}

    \begin{algorithm}
        \caption{Huffman's Algorithm for Ternary Codes}
        \KwIn{$f : f[1 \dots n]$}
        \KwOut{$T : $ a tree with $n$ leaves}

        $T$: empty tree;
        
        $H$: priority queue ordered by $f$;
        
        \For{$i := 1 \textbf{ in } n$}{
            insert($H$, $i$);
        }

        \For{$k := n + 1 \textbf{ in } \lceil 2n-n/2 \rceil$}{
            i := extract\_min($H$);

            j := extract\_min($H$);

            k := extract\_min($H$);

            Create a node $r$ in $T$ with children $i, j, k$;

            $f[r] = f[i] + f[j] + f[k];$

            insert($H$, $r$);
        }

        \Return{T}
    \end{algorithm}

    Claim: Huffman's algorithm for ternary codes yields an optimal ternary code. Every optimal solution has three least frequent symbols as leaves connected to an internal node of greatest depth

    Proof: Let $T$ be an optimal ternary code. Let $i, j, k$ be the three least frequent symbols. Let $r$ be the internal node of greatest depth. Let $T'$ be the tree obtained from $T$ by removing $i, j, k$ and $r$. Let $T''$ be the tree obtained from $T$ by removing $r$. Then $T'$ is an optimal ternary code and $T''$ is an optimal binary code.

    \newpage

    \question[4]{}

    For sake of contradiction, assume that the greedy algorithm is not optimal. Given a solution provided by the greedy algorithm represented in a numbered set $T$ of size $n$. So a package $p_i \in T$ where $i$ is the indexed number of the package. Each package has a weight $w_i$ and a corresponding truck $t_j$ that it is assigned to be on. 
    
    Since we know that the greedy algorithm is not optimal, there must be an optimal solution $T'$. Assume that $T'$ has $m$ trucks and assume that $T$ has $n$ trucks. We can say that $m \leq n$. For this proof, we will assume that $T' \in T$ such that $T'$ is a subset of $T$. 

    Let's say that truck $t'_m$ is the last truck in $T'$. If $t'_m > t_m$, then we arrive at a contradiction because the greedy solution utilized more trucks than the optimal solution, which is not valid. If $t'_m < t_m$, then we can say that the greedy solution utilized less trucks than the optimal solution, which is also not valid. 
    
    Thus, we can conclude that the greedy solution is the optimal solution.

\end{document}
